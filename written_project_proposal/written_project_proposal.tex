\documentclass{article}

% Paper size and margins
\usepackage[letterpaper,margin=0.5in]{geometry}

% Use spaces between paragraphs
\usepackage{parskip}

% No numbers in section titles
\usepackage{titlesec}
\titlelabel{}

% Bibliography
\usepackage[style=ieee]{biblatex}
\bibliography{written_project_proposal}

\title{Towards a Verified Bzip2: A Functional Specification for Burrows-Wheeler}
\author{Jake Waksbaum '20\\Advisor: Andrew Appel}
\date{}

\begin{document}
\maketitle

\section{Motivation and Goal}
File compression software is often used to reduce the amount of disk
space occupied by archived files and to more efficiently transfer
data. Many types of file compression software, such as
bzip2\cite{bzip2}, 7zip\cite{7zip}, unzip\cite{infozip}, and
WinRAR\cite{winrar} have been found vulnerable to security exploits
that can be used in denial-of-service attacks or to enable arbitrary
code execution. Often these vulnerabilities are the result of
programmer errors related to the management of low-level resources:
buffer-overflow, use-after-free, double-free, etc. These types of
errors are well-addressed using formal verification methods that
demand proof that low-level resources are correctly managed.

The goal of my project is to begin the work of producing a formally
verified implementation of the popular and open-source bzip2
compression algorithm. A formally verified compression and
decompression program won't suffer from the types of security exploits
common to existing implementations.

\section{Problem Background and Related Work}
Coq is a theorem proving assistant that allows programmers to write
machine-checked proofs: proofs are verified by Coq to be valid. It
also includes a functional programming language called Gallina that
can be used to produce functional specifications\cite{7536361} that
are useful for describing the desired behavior of some C program.

The CompCert
compiler\cite{Stewart:2015:CC:2676726.2676985,leroy:inria-00000963} is
an optimizing C compiler, that has been formally verified in Coq to
preserve semantics: this means that the semantics of the executable
code with respect to the semantics of the assembly language are the
same as the semantics of the source code with respect to the semantics
of the source language, C. This allows us to have confidence that the
properties that we prove about our C programs also hold on the
compiled executables.

The Verified Software
Toolchain\cite{Appel:2011:VST:1987211.1987212,Appel:2014:PLC:2670099}
includes also a program logic for C called Verifiable C that is proved
sound with respect to the operational semantics of CompCert C. This
gives us tools in Coq to reason about our C programs and to prove
properties such as correctness and safety. It includes
VST-Floyd\cite{cao2018vst-floyd:}, which helps automate some parts of
the process of building functional correctness proofs of C programs.

These tools were used to formalize the functional specifications of
the mbedTLS HMAC-DRBG, to prove that the specification truly produces
pseuodrandom output, and then to produce an executable for which that
result also holds\cite{Ye:2017:VCS:3133956.3133974}.

\section{Approach}
I plan to work towards a formally verified implementation of bzip2 by
implementing functional specifications for the core compression and
decompression passes that make up the bzip2 stack: run-length
encoding, Burrows-Wheeler, Huffman encoding, etc. This means writing a
functional program in Gallina that implements those algorithms, and
proving useful properties about those algorithms such as that the
decoder inverts the encoder. I will focus on the functional
specification of the algorithms as opposed to the verification of C
implementations.

\section{Plan}
I will start by working on a functional specification for run-length
encoding, as it is simple and the first phase of compression. Then, I
will work on a functional specification for Burrows-Wheeler. Depending
on how long that takes, I can move on to the rest of the compression
stack, implementing and proving theorems about as much of it as I can
in the time that I have. Each component of the stack is easy to treat
separately and can be proved correct independent of the other
components.

\section{Evaluation}
My main goal is to produce a functional specification for
Burrows-Wheeler and a proof that the decoder inverts the encoder. A
stretch goal would be to verify the entire bzip2 compression stack,
and an even bigger stretch goal would be to produce an entire Coq
program that can operate on real bzip2 zip files. Evaluating whether
or not I have completed the proof is pretty simple: Coq will reject
any incorrect proofs of my theorems.

\printbibliography{}

\end{document}
