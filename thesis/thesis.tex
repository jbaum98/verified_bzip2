\documentclass[11pt]{article}

\usepackage{graphicx}
\usepackage{grffile}
\usepackage{longtable}
\usepackage{wrapfig}
\usepackage{rotating}
\usepackage[normalem]{ulem}
\usepackage{amsmath}
\usepackage{textcomp}
\usepackage{amssymb}
\usepackage{capt-of}
\usepackage{hyperref}

\usepackage{amsthm}
\usepackage{thmtools}
\declaretheorem[style=plain,numberwithin=section]{theorem}
\declaretheorem[style=plain,sibling=theorem]{proposition}
\declaretheorem[style=plain,sibling=theorem]{lemma}
\declaretheorem[style=plain,sibling=theorem]{corollary}

\declaretheorem[style=definition,numberwithin=section]{definition}
\declaretheorem[style=definition,numberwithin=section]{example}
\declaretheorem[style=definition,numberwithin=section]{remark}

\setlength{\parskip}{1em}
\setlength{\parindent}{0em}
\usepackage[letterpaper,margin=1in]{geometry}

\author{Jake Waksbaum}
\date{\today}
\title{Thesis}
\hypersetup{
 pdfauthor={Jake Waksbaum},
 pdftitle={Thesis},
 pdfkeywords={},
 pdfsubject={},
 pdfcreator={Emacs 26.1 (Org mode 9.1.9)},
 pdflang={English}}

\begin{document}

\maketitle
\tableofcontents

\section{Introduction}
\label{sec:intro}

The Burrows-Wheeler transform (BWT) is a reversible transformation
that makes a string more amenable to compression by other
methods \cite{bw}. The forwards-transform is straightforward to
describe: for a string of length \(n\), form the \(n \times n\) matrix of
all cyclic shifts of the original string, sort it lexicographically,
and take the last column of the matrix. In addition, we note that
index where the original string appears in the sorted rotation matrix.
As shown in \autoref{fig:bw_ex}, applying the algorithm to the string
\verb|abracadabra!| produces the new string \verb|ard!rcaaabb| and the
index \verb|3|. The tendency of the transformation to group the same
character together is what makes the resulting restring easier to
compress. Although it is not at all obvious, the transformed string,
together with the index, provide enough information to recover the the
original string. BWT is discussed more in \autoref{sec:bwt}.

\begin{figure}
  \centering
  \begin{tabular}{rc}
    0  & abracadabra! \\
    1  & bracadabra!a \\
    2  & racadabra!ab \\
    3  & acadabra!abr \\
    4  & cadabra!abra \\
    5  & adabra!abrac \\
    6  & dabra!abraca \\
    7  & abra!abracad \\
    8  & bra!abracada \\
    9  & ra!abracadab \\
    10 & a!abracadabr \\
    11 & !abracadabra
  \end{tabular}
  $\rightarrow$
  \begin{tabular}{rc}
    0  & !abracadabr\textbf{a} \\
    1  & a!abracadab\textbf{r} \\
    2  & abra!abraca\textbf{d} \\
    \textit{3} & \textit{abracadabra\textbf{!}} \\
    4  & acadabra!ab\textbf{r} \\
    5  & adabra!abra\textbf{c} \\
    6  & bra!abracad\textbf{a} \\
    7  & bracadabra!\textbf{a} \\
    8  & cadabra!abr\textbf{a} \\
    9  & dabra!abrac\textbf{a} \\
    10 & ra!abracada\textbf{b} \\
    11 & racadabra!a\textbf{b}
  \end{tabular}
  \caption{BWT acting on \texttt{abracadabra!}}
  \label{fig:bw_ex}
\end{figure}

Bird and Mu \cite{birdmu,pearls} analyze the Burrows-Wheeler algorithm
in a functional setting by expressing the algorithm as a composition
of many smaller functions. Then, they derive an expression for the
inverse-transform from the specification that it invert the
forwards-transform. In addition to providing an intuitive derivation
of the inverse-transform, this provides the outline of a proof of the
correctness of the Burrows-Wheeler algorithm. We give our version of
their derivation in \autoref{sec:deriv}

Our contribution is a formalization in the Coq proof assistant of the
derivations in \cite{birdmu,pearls}. This is useful in its own right
as a functional specification for a potential verified implementation
of a program that relies on Burrows-Wheeler \cite{appel-func-spec}.
More generally, this provides evidence that Bird's approach of
program calculation, in which programs are derived from their
specifications via algebraic manipulations, is useful for proving the
correctness of algorithms in a machine-checked setting. Bird has
applied this approach to greedy algorithms, the Boyer-Moore algorithm,
the Knuth-Morris-Pratt algoirthm, Sudoku solvers, arithmetic coding,
and the Schorr-Waite algorithm, the Johnson-Trotter algorithm, and
many more problems throughout his career \cite{pearls}. Finally, as a
part of this proof we develop a theory of stable sorts, proving
equivalence between different definitions of sorted lists and stable
sorts, as well as the equivalence of all stable sort algorithms to
each other.

\section{Burrows-Wheeler}
\label{sec:bw}

The Burrows-Wheeler transform was originally introduced as a pass in a
compression algorithm. First the input is transformed with BWT, which
tends to group together characters. The intuition for why this happens
is that all adjacent characters in the output string are followed in
the input string by characters that are adjacent in the sorted first
column. If there is any correlation between a character and the
character preceeding it, bringing together the characters in the first
column will tend to bring together similar characters in the last
column.

The next pass is a Move-to-Front (MTF) encoding, in which a character
is represented by the number of different characters encountered since
the last occurence of this character. This creates many runs of zeroes
from the runs in the output of BWT. On English texts, often 50-60\% of
the values produced by the MTF pass are zeroes
\cite{fenwick2007,bw-analysis}. Often a Run-Length Encoding (RLE) pass
is then used to encode only the runs of zeroes with their lengths,
using a bijective base-2 encoding \cite{bw-analysis, tsai_2016}.
Finally, some sort of zeroth-order coder like a Huffman coder is used.

BWT is also at the heart of algorithms that exploit the relationship
between the rotation matrix in BWT and the suffix array data structure
in order to search for patterns in a compressed text
\cite{ferragina_index}.

\section{The Derivation}
\label{sec:deriv}
In this section, we give an overview of the derivation without delving
into the proofs of the lemmas. We only derive the simples version of
the algorithm, although the derivations in \cite{birdmu,pearls}
provide additional optimizations that bring the final implementation
closer to the imperative implementation in \cite{bw} and other texts.

\subsection{Forwards-Transform}
\label{subsec:ft}
In general, the Burrows-Wheeler transform can be applied to lists of
any type that can be sorted, so we define our functions to operate on
\verb|list A|. We will discuss the sortable constraint in
Section~\autoref{sec:sort_stable}.

First, we define left and right rotations of a list.
\begin{verbatim}
Definition lrot (l: list A) : list A :=
  match l with
  | [] => []
  | hd :: tl => tl ++ [hd]
  end.

Definition rrot (l: list A) : list A :=
  match l with
  | [] => []
  | hd :: tl => last l hd :: init l
  end.
\end{verbatim}

Now we can produce the rotation matrix by repeatedly applying \verb|lrot|.
\begin{verbatim}
Fixpoint iter (f : A -> A) (n : nat) (z : A) : list A :=
  match n with
  | O   => []
  | S m => z :: iter f m (f z)
  end.

Definition rots (l : list A) : list (list A) :=
  iter lrot (length l) l.
\end{verbatim}

Finally, we produce the transformed string by taking the last column
of the sorted rotation matrix. \verb|lexsort| sorts the rotation matrix
lexicographically; it's precise implementation is not important.
\begin{verbatim}
Definition bwp (l: list A) : list A :=
  match l with
  | [] => []
  | hd :: _ => List.map (fun x => last x hd) (lexsort (rots l))
  end.
\end{verbatim}

We also need to find the index in the sorted rotation matrix where the
original string appears.
\begin{verbatim}
Fixpoint findIndex (x : A) (ls : list A) : nat :=
  match ls with
  | [] => 0
  | hd :: tl =>
    match x == hd with
    | left _ => 0
    | right _ => S (findIndex x tl)
    end
  end.
\end{verbatim}
\begin{verbatim}
Definition bwn (l : list A) : nat :=
  findIndex l (lexsort (rots l)).
\end{verbatim}

Without proof, we note that we should have for any nonempty
\verb|x : list A| and \verb|d : A|,
\begin{verbatim}
nth (bwn x) (lexsort (rots x)) d = x
\end{verbatim}

\subsection{Inverse-Transform}
\label{subsec:invt}

Now we have to construct a function \verb|unbwt| such that
\begin{verbatim}
unbwt (bwn xs) (bwp xs) = xs.
\end{verbatim}
If we could recreate the entire sorted rotation matrix from
\verb|bwp x|, it woiuld be simple to recover \verb|x| by indexing into
the matrix at \verb|bwn x|. That is, we would like a function
\verb|recreate| such that
\begin{verbatim}
recreate (bwp l) = lexsort (rots l).
\end{verbatim}

Then, we could define \verb|unbwt| as
\begin{verbatim}
Definition unbwt (i : nat) (l : list A) : list A :=
  nth i (recreate l) l.
\end{verbatim}
and we would have the correctness of \verb|unbwt| as a result of the
correctness of \verb|bwn|.

\subsubsection{\texttt{recreate}}
We will recreate the sorted rotation matrix column by column, so
\verb|recreate j| will recreate the first \verb|j| columns of the sorted
rotation matrix.
\begin{verbatim}
Definition cols (j : nat) (l : list A) : list A := map (firstn j) l.

Theorem recreate_correct_ind : forall j l,
    j <= length l ->
    recreate j (bwp l) = cols j (lexsort (rots l)).
\end{verbatim}

To derive an implementation of \verb|recreate| from this correctness
specification, we need an expression for \verb|recreate 0| and
\verb|recreate (S j)|. We can substitute for \verb|j| on the
right-hand side and manipulate that expression algebraically until it
is in the form \verb|_ (bwp l)|. Then we can define \verb|recreate 0|
and \verb|recreate (S j)| to be precisely the \verb|_|, and by
running our derivation in reverse, we obtain a proof for the base and
inductive cases that uses only rewrites.

We start with the base case, \verb|recreate 0|:
\begin{verbatim}
  cols 0 (lexsort (rots l))
= { by unfold cols }
  map (firstn 0) (lexsort (rots l))
= { by firstn_O}
  map (const []) (lexsort (rots l))
= { by map_const }
  repeat [] (length (lexsort (rots l)))
= { by bwp_length, rots_length, lexsort_length }
  repeat [] (length (bwp l))
= { by map_const }
  map (const []) (bwp l)
\end{verbatim}

From this, we can define \verb|recreate 0 l = map (const []) l| and
we already have the proof for the base case of the correctness
specification.

For the inductive case, we have the expression
\begin{verbatim}
  cols (S j) (lexsort (rots l))
\end{verbatim}
on the right-hand side. Our plan is to express \verb|cols (S j)|
in terms of \verb|cols j|, and then to use the inductive hypothesis
\begin{verbatim}
  cols j (lexsort (rots l)) = recreate j
\end{verbatim}
to produce a well-formed recursive definition.

One way to do this is to say that taking the first \verb|S j| columns
of a matrix is the same as moving the first column to the back, taking
the first \verb|j| columns, and then sticking the last column back on
the front:
\begin{verbatim}
  cols (S j) l = prepend_col (map last (map lrot l)) (cols j (map lrot l))
\end{verbatim}
If we substitute \verb|map rrot l| for \verb|l| we find a simpler
expression
\begin{verbatim}
  cols (S j) (map rrot l) = prepend_col (map last l) (cols j l)
\end{verbatim}

Now we would like to find a way of introducing a \verb|map rrot|
before the \verb|lexsort (rots l)|. The insight here is that we can
assume that \verb|lexsort| is actually a radix-sort. That is, that
\begin{verbatim}
lexsort xs = rep (length xs) (hdsort ∘ map rrot) xs
\end{verbatim}
where \verb|hdsort| sorts the matrix stably on the the first column
and \verb|rep| is defined as
\begin{verbatim}
Fixpoint rep (f : A -> A) (n : nat) (z : A) : A :=
  match n with
  | O => z
  | S m => f (rep f m z)
  end.
\end{verbatim}

We can do this because there is only one way to sort the matrix, so
assuming \verb|lexsort| and the radix-sort are correct, they must be
doing the same thing. We will specify precisely when and why this is
valid later.

Stepping back, it is important to say that this insight, whether
explicitly stated or not, is always the crucial step in inverting the
Burrows-Wheeler transform. Because radix-sort only needs one column at
a time, this transformation will allow us to reconstruct the sorted
rotation matrix one column at a time.

In seeking to introduce an extra \verb|map rrot| term, we can note
that rotating and sorting an extra time doesn't change anything when
we are sorting the rotation matrix. This works because rotating the
last column to the front produces a matrix that is sorted on its last
\verb|n - 1| characters, so sorting stably on the first column will
re-sort the matrix. Crucially though, rotating the last column
forwards also permutes the entire matrix, so in addition to having a
sorted list, we have a sorted list of the same elements.

More precisely, we can characterize how \verb|map rrot| permutes
\verb|rots|
\begin{verbatim}
map rrot (rots x) = rrot (rots x)
\end{verbatim}

Using that we can show
\begin{verbatim}
  lexsort (rots x)
= { because rrot permutes its elements }
  lexsort (rrot (rots x))
= { because hdsort permutes its elements }
  lexsort (hdsort (rrot (rots x)))
= { radix-sort }
  rep n (hdsort ∘ rrot) (hdsort (rrot (rots x)))
= { rep_absorb_r }
  rep (S n) (hdsort ∘ rrot) (rots x)
= { rep_split_l }
  hdsort (rrot (rep n (hdsort ∘ rrot) (rots x)))
= { radix-sort }
  hdsort (rrot (lexsort (rots x)))
\end{verbatim}

From here we can calculate
\begin{verbatim}
  cols (S j) (lexsort (rots l))
= cols (S j) (hdsort (map rrot (lexsort (rots l))))
= { hdsort commutes with cols (S j) }
  hdsort (cols (S j) (map rrot (lexsort (rots l))))
= { reduce cols (S j) }
  hdsort (prepend_col (map last (lexsort (rots l))) (cols j (lexsort (rots l)))
= { by IH and def of bwp }
  hdsort (prepend_col (bwp l) (recreate j (bwp l)))
= (fun x => hdsort (prepend_col x (recreate j x))) bwp
\end{verbatim}

We can use this to define \verb|recreate|:
\begin{verbatim}
Fixpoint recreate (j : nat) (l : list A) : list (list A) :=
  match j with
  | O    => map (const []) l
  | S j' => hdsort (prepend_col l (recreate j' l))
  end.
\end{verbatim}

\section{Proving the Lemmas}
\label{sec:lemmas}

In this section we prove the lemmas from the previous section:
\begin{enumerate}
\item
\begin{verbatim}
map rrot (rots x) = rrot (rots x)
\end{verbatim}

\item \verb|hdsort| commutes with \verb|cols (S j)|
  This proof is cool because in theory it can be proven via
  parametricity, but that's a metatheorem in Coq.

\item
\begin{verbatim}
cols (S j) (map rrot l) = prepend_col (map last l) (cols j l)
\end{verbatim}
\end{enumerate}

\section{Sorting \& Stability}
\label{sec:sort_stable}

\subsection{Sorting}
\label{subsec:sorting}
What do we mean by the constraint that \verb|A| is sortable? We say
that a type must be equppied with a total, decidable preorder in order
to be sorted.
\begin{verbatim}
Class Preord (A : Type):=
  { le : A -> A -> Prop;
    le_trans : forall x y z, le x y -> le y z -> le x z;
    le_total : forall x y, le x y \/ le y x;
    le_dec : forall x y, {le x y} + {~le x y};
  }.
\end{verbatim}
The type of a sorting algorithm will be
\begin{verbatim}
sort : forall (A : Type), Preord A -> list A -> list A
\end{verbatim}

A correct sorting algorithm should satisfy
\begin{verbatim}
forall x, Sorted (sort x) /\ Permutation x (sort x)
\end{verbatim}
where we define the \verb|Sorted| relation on lists:
\begin{verbatim}
Inductive Sorted: list A -> Prop :=
| Sorted_nil:
    Sorted nil
| Sorted_cons: forall hd tl,
    (forall x, In x tl -> le hd x) ->
    Sorted tl ->
    Sorted (hd :: tl).
\end{verbatim}

We can also have defined \verb|Sorted| like this
\begin{verbatim}
Inductive LocallySorted : list A -> Prop :=
| LSorted_nil : LocallySorted nil
| LSorted_cons1 a : LocallySorted (a :: nil)
| LSorted_consn a b l :
    LocallySorted (b :: l) -> le a b -> LocallySorted (a :: b :: l).
\end{verbatim}

\begin{theorem}[Sorted-LocallySorted Equivalent]
\begin{verbatim}
forall l, Sorted l <-> LocallySorted l.
\end{verbatim}
\end{theorem}

\begin{proof}
  First we prove the forwards direction by induction on the evidence
  \verb|Sorted l|. For the base case, we have \verb|Lsorted_nil|. In
  the inductive case, we consider when \verb|tl| is empty and when it
  is \verb|b :: tl|. If it is empty, we have \verb|Lsorted_cons1 hd|.
  If it is not, we have \verb|LocallySorted (b :: tl)| by the
  inductive hypothesis. We also have that
  \verb|forall x, In x (b :: tl) -> le hd x)|, and because clearly
  \verb|In b (b :: tl)| we have \verb|le hd b|. This means we can
  apply \verb|Lsorted_consn| and we are done.

  Now we prove the backwards direction by induction on the evidence
  \verb|LocallySorted l|. The nil case is given by \verb|Sorted_nil|,
  and the singleton case is given by the \verb|Sorted_cons| followed
  by \verb|Sorted_nil|, using the fact that \verb|In x []| is
  \verb|False|. In the inductive case we have \verb|Sorted (b :: tl)|
  by the inductive hypothesis, which implies \verb|Sorted tl|. It
  suffices to show \verb|forall x, In x (b :: tl) -> le a x|.
  \verb|Sorted (b :: tl)| implies that
  \verb|forall x, In x tl -> le b x|, and we have that \verb|le a b|,
  so by the transitive property of \verb|le| we are done.
\end{proof}

We could also define sorted based on indexes:
\begin{verbatim}
Definition IndexSorted (l : list A) := forall i j d,
    i <= j < length l ->
    le (nth i l d) (nth j l d).
\end{verbatim}

\begin{theorem}[Sorted-IndexSorted Equivalent]
\begin{verbatim}
forall l, Sorted l <-> IndexSorted l.
\end{verbatim}
\end{theorem}

\begin{theorem}[Permutation-IndexPermutation Equivalent]
  The built-in \verb|Permutation xs ys| relation is equivalent to saying
  that there exists some permutation function \verb|p : 'S_n| such that
\begin{verbatim}
forall i : 'I_n, nth i xs = nth (p i) ys
\end{verbatim}
\end{theorem}

\subsection{Stability}
Any preorder induces a decidable equivalence relation:
\begin{verbatim}
Definition eqv {A} `{Preord A} (x y : A) : Prop := le x y /\ le y x.
\end{verbatim}
A stable sort is a sort that preserves the relative order of
equivalent elements. We can state this naturally in terms of the
permutations. We say that a permutation \verb|p| is stable on a list
\verb|l| if
\begin{verbatim}
forall i j : 'I_n, i < j -> eqv (nth i l) (nth j l) -> p i < p j
\end{verbatim}
Then, a sort is stable if for all \verb|l|, there exists a stable
permutation between \verb|l| and \verb|sort l|.

We can also say that two lists are stably related if
\begin{verbatim}
forall x, filter (eqv_dec x) l = filter (eqv_dec x) l'
\end{verbatim}

\begin{theorem}[PermStable-FilterStable Equivalent]
\begin{verbatim}
forall l,
  PermStable l l' <-> (FilterStable l l' /\ Permutation l l')
\end{verbatim}
\end{theorem}

Stability is important because we can prove that there is only one way
to stably sort any list. This also means that any two correct stable
sorting algorithms must agree always, and can be substituted for one another.

\begin{theorem}[StableUnique]
\begin{verbatim}
forall l l',
  Sorted l' -> Permutation l l' -> Stable l l' -> l = l'
\end{verbatim}
\end{theorem}

If we have a antisymmetric total preorder, also known simply as a
total order, then we have the important result that all sorts are
stable sorts. Antisymmetricity means that
\begin{equation*}
  \forall x y, x \leq y \wedge y \leq x \implies x = y
\end{equation*}
or in terms of equivalence
\begin{equation*}
  \forall x y, x \equiv y \implies x = y
\end{equation*}

\begin{theorem}[OrderStable]
\begin{verbatim}
forall (O : Preorder A) (l l' : A),
  (forall x y, eqv x y -> x = y) ->
  Permutation l l' -> Sorted l -> Sorted l' -> Stable l l'
\end{verbatim}
\end{theorem}

We require that the type \verb|A| be equipped with a decidable total
order. If \verb|A| has a decidable total order, then the lexicographic
preorder on \verb|list A| is also an order. This means that any
sorting algorithm using lexicographic order is stable, including
radix-sort. This is what allows us to conclude that \verb|lexsort| is
equivalent to radix-sort.

\begin{theorem}[RadixSort Correct]
\begin{verbatim}
forall x,
  Sorted (radixsort x) /\ Permutation x (radixsort x)
\end{verbatim}
\end{theorem}

\nocite{*}

\bibliography{thesis}{}

\bibliographystyle{plain}

\end{document}
