\documentclass{beamer}
\usetheme{metropolis}
\usepackage{pgfplots}
\usepackage{ulem}
\usepackage{appendixnumberbeamer}
\usepackage{proof}

\newcommand\doubleplus{\ensuremath{\mathbin{+\mkern-3mu+}}}

\usepackage{pgfpages}
%\setbeameroption{show notes on second screen=bottom}

\makeatletter
\def\beamer@framenotesbegin{% at beginning of slide
     \usebeamercolor[fg]{normal text}
      \gdef\beamer@noteitems{}%
      \gdef\beamer@notes{}%
}
\makeatother

\title{Towards a Verified Bzip2}
\subtitle{A Functional Specification for Burrows-Wheeler}
\date{April 23, 2019}
\author{Jake Waksbaum '20\\Advisor: Prof. Andrew Appel}
\institute{Princeton University}

\setmonofont{Iosevka}

\metroset{block=fill}

\pgfplotsset{compat=1.16}
\pgfplotsset{
  barplot/.style={
    %ybar stacked,
    %symbolic x coords={bzip2, 7zip, winRAR, unzip},
    % xtick=data,
    cycle list name=mbarplot cycle,
    axis y line*=none,
    legend style={
      draw=none,
      fill=none,
      cells={anchor=west},
      anchor={north west},
      % at={(0.02,0.98)},
    },
    xmajorgrids=false,
    ymajorgrids=true,
    x tick label style={
      font=\footnotesize
    },
    y tick label style={
      font=\footnotesize
    },
    legend style={
      font=\footnotesize
    },
    major grid style={
      dotted,
    },
    axis x line*=bottom,
  }
}

\begin{document}
  \maketitle

  \begin{frame}{Overview}
    \tableofcontents
  \end{frame}

  \section{Background: The Burrows-Wheeler Transform}

  \begin{frame}{The Burrows-Wheeler Transform}
    \textit{A Block-sorting Lossless Data Compression Algorithm} (1994)
    \begin{itemize}
    \item<+-> A transformation of type $list~A \to (list~A, \mathbb{N})$
    \item<+-> Makes it easier to compress
    \item<+-> \textbf{Reversible}
    \end{itemize}
  \end{frame}

  \begin{frame}{The Transform}
    Input text: \texttt{abracadabra!}
    \begin{columns}[c] % the "c" option specifies center vertical alignment
      \column{.5\textwidth}
      \begin{enumerate}
      \item<2-> Form rotation matrix
      \item<3-> Sort rotation matrix
      \item<4-> Take the last column of the sorted rotation matrix\ldots
        \only<5->{ and the index where the original string ends up}
      \end{enumerate}

      \column{.5\textwidth}
      \only<2>{%
        \begin{tabular}{rc}
          0  & abracadabra! \\
          1  & bracadabra!a \\
          2  & racadabra!ab \\
          3  & acadabra!abr \\
          4  & cadabra!abra \\
          5  & adabra!abrac \\
          6  & dabra!abraca \\
          7  & abra!abracad \\
          8  & bra!abracada \\
          9  & ra!abracadab \\
          10 & a!abracadabr \\
          11 & !abracadabra
        \end{tabular}
      }%
      \only<3->{%
        \begin{tabular}{rc}
          0  & !abracadabr\alert<4->{a} \\
          1  & a!abracadab\alert<4->{r} \\
          2  & abra!abraca\alert<4->{d} \\
          \textbf<5->{3} & \textbf<5->{abracadabra\alert<4->{!}} \\
          4  & acadabra!ab\alert<4->{r} \\
          5  & adabra!abra\alert<4->{c} \\
          6  & bra!abracad\alert<4->{a} \\
          7  & bracadabra!\alert<4->{a} \\
          8  & cadabra!abr\alert<4->{a} \\
          9  & dabra!abrac\alert<4->{a} \\
          10 & ra!abracada\alert<4->{b} \\
          11 & racadabra!a\alert<4->{b}
        \end{tabular}
      }%
    \end{columns}
  \end{frame}

  \begin{frame}{Things that are not clear}
    \begin{itemize}
    \item<+-> Why we would want to do this
    \item<+-> How we undo this
    \end{itemize}
  \end{frame}

  \section{Contributions}
  \begin{frame}{}
    \begin{enumerate}
    \item<+-> Implemented the BWT and the reverse transform in
      \alert<4->{Coq}, and produced a \alert<4->{machine checked}
      proof of correctness
      \begin{equation*}
        \forall~l: list~A, unbwt~(bwt~l) = l
      \end{equation*}
    \item<+-> Implemented and produced a \alert<4->{machine checked} proof
      of correctness for radix sort.
    \item<+-> Explored different definitions of \textbf{sortedness},
      \textbf{permutation}, and \textbf{stability} and proved
      equivalence between them.
    \end{enumerate}
  \end{frame}

  \foreach \X in {1,...,10} {%
    \begin{frame}{Coq}
      \centering
      \includegraphics[width=\textwidth]{img/coq-example-\X.png}
    \end{frame}%
  }

  \section{Proving Radixsort Correct}

  \begin{frame}{Radixsort}{}
    \begin{center}
      \begin{tt}
        \only<1-2>{%
          \begin{tabular}{c}
            d a \alert<2>{b} \\
            a d \alert<2>{d} \\
            c a \alert<2>{b} \\
            f a \alert<2>{d} \\
            f e \alert<2>{e} \\
            b a \alert<2>{d} \\
            d a \alert<2>{d} \\
            b e \alert<2>{e} \\
            f e \alert<2>{d} \\
            b e \alert<2>{d} \\
            e b \alert<2>{b} \\
            a c \alert<2>{e}
          \end{tabular}
        }%
        \only<3-4>{%
          \begin{tabular}{c}
            d \alert<4>{a} \alert<3>{b} \\
            c \alert<4>{a} \alert<3>{b} \\
            e \alert<4>{b} \alert<3>{b} \\
            a \alert<4>{d} \alert<3>{d} \\
            f \alert<4>{a} \alert<3>{d} \\
            b \alert<4>{a} \alert<3>{d} \\
            d \alert<4>{a} \alert<3>{d} \\
            f \alert<4>{e} \alert<3>{d} \\
            b \alert<4>{e} \alert<3>{d} \\
            f \alert<4>{e} \alert<3>{e} \\
            b \alert<4>{e} \alert<3>{e} \\
            a \alert<4>{c} \alert<3>{e}
          \end{tabular}
        }%
        \only<5-6>{%
          \begin{tabular}{c}
            \alert<6>{d} \alert<5>{a} b \\
            \alert<6>{c} \alert<5>{a} b \\
            \alert<6>{f} \alert<5>{a} d \\
            \alert<6>{b} \alert<5>{a} d \\
            \alert<6>{d} \alert<5>{a} d \\
            \alert<6>{e} \alert<5>{b} b \\
            \alert<6>{a} \alert<5>{c} e \\
            \alert<6>{a} \alert<5>{d} d \\
            \alert<6>{f} \alert<5>{e} d \\
            \alert<6>{b} \alert<5>{e} d \\
            \alert<6>{f} \alert<5>{e} e \\
            \alert<6>{b} \alert<5>{e} e
          \end{tabular}
        }%
        \only<7-8>{%
          \begin{tabular}{c}
            \alert<7>{a} c e \\
            \alert<7>{a} d d \\
            \alert<7>{b} a d \\
            \alert<7>{b} e d \\
            \alert<7>{b} e e \\
            \alert<7>{c} a b \\
            \alert<7>{d} a b \\
            \alert<7>{d} a d \\
            \alert<7>{e} b b \\
            \alert<7>{f} a d \\
            \alert<7>{f} e d \\
            \alert<7>{f} e e
          \end{tabular}
        }%
      \end{tt}
    \end{center}
  \end{frame}

  \begin{frame}{Radixsort -- Questions}
    \begin{enumerate}
    \item<+-> What does it mean for radixsort to be \textit{correct}?

      \onslide<+->{%
        A function $f$ is a sorting function if for all lists $l$
        \begin{equation*}
          \alert<5>{Sorted}~(f~l) \land \alert<5>{Permutation}~(f~l)~l
        \end{equation*}
      }%
    \item<+-> Radixsort relies on its inner sort being \textit{stable}

      \onslide<+->{%
        A function $f$ is a \textit{stable} sorting function if for all lists $l$
        \begin{equation*}
          \alert<5>{Sorted}~(f~l) \land \alert<5>{StablePermutation}~(f~l)~l
        \end{equation*}
      }%
    \end{enumerate}
    \onslide<+->{We want to define these properties}
  \end{frame}

  \begin{frame}{Definitions Matter -- $Sorted$}
    \only<+>{%
      \begin{block}{Index Definition: $SortedIx$}
        \begin{equation*}
          i < j \to l_i < l_j
        \end{equation*}
      \end{block}
      \bigskip
      \begin{block}{Inductive Definition: $Sorted$}
        \begin{equation*}
          \infer[SortedNil]{Sorted~[]}{}
        \end{equation*}

        \begin{equation*}
          \infer[SortedCons]{Sorted~(a :: l)}{%
            (\forall~x \in l,~a < x) & Sorted~l
          }
        \end{equation*}
      \end{block}
    }%
    \only<+>{%
      \bigskip
      \begin{block}{Inductive Definition II: $LocalSorted$}
        \begin{equation*}
          \infer[LocalSortedNil]{LocalSorted~[]}{}
        \end{equation*}

        \begin{equation*}
          \infer[LocalSortedSingle]{LocalSorted~[a]}{}
        \end{equation*}

        \begin{equation*}
          \infer[LocalSortedCons]{LocalSorted~(a :: b :: l)}{%
            a < b & LocalSorted~(b :: l)
          }
        \end{equation*}
      \end{block}
    }%
    \only<+>{%
      $Sorted~l \iff SortedIx~l \iff LocalSorted~l$
      \begin{itemize}
      \item We can choose most convenient definition for a proof
      \item Gives us confidence in our definitions
      \end{itemize}
    }%
  \end{frame}

  \begin{frame}{$Permutation$}
    \only<+>{%
      \begin{block}{Index Definition: $PermutationIx$}%
        \setlength{\abovedisplayskip}{\abovedisplayshortskip}
        \setlength{\belowdisplayskip}{\belowdisplayshortskip}
        \begin{equation*}
          \exists~p : list~\mathbb{N}, perm~(length~l)~p \land apply~p~l = l'
        \end{equation*}
        where $perm~n~p$ means that for all $i$
        \begin{equation*}
          i < n \iff i \in p
        \end{equation*}
      \end{block}
      \bigskip
      \begin{block}{Counting Definition: $PermutationCount$}
        \begin{equation*}
          \forall~x, count~x~l = count~x~l'
        \end{equation*}
      \end{block}
    }
    \only<+>{%
      \setlength{\abovedisplayskip}{\abovedisplayshortskip}
      \setlength{\belowdisplayskip}{\belowdisplayshortskip}
      \begin{block}{Inductive Definition: $Permutation$}
        \begin{equation*}
          \infer[PermutationNil]{Permutation~[]~[]}{}
        \end{equation*}

        \begin{equation*}
          \infer[PermutationSkip]{Permutation~(a :: l)~(a :: l')}{%
            Permutation~l~l'
          }%
        \end{equation*}

        \begin{equation*}
          \infer[PermutationSwap]{Permutation~(x :: y :: l)~(y :: x :: l)}{}
        \end{equation*}

        \begin{equation*}
          \infer[PermutationTrans]{Permutation~l~l''}{%
            Permutation~l~l' & Permutation~l'~l''
          }%
        \end{equation*}
      \end{block}
    }
  \end{frame}

  \begin{frame}{$StablePermutation$}
    \only<1-2>{%
      \begin{block}{Index Definition: $StablePermImage$ and $StablePermPreimage$}
        \setlength{\abovedisplayskip}{\abovedisplayshortskip}
        \setlength{\belowdisplayskip}{\belowdisplayshortskip}
        \begin{equation*}%
          \exists~p : list~\mathbb{N}, stable~l~p
          \land perm~(length~l)~p \land apply~p~l = l'
        \end{equation*}
        where $stable~l~p$ means that for all $i < j$
        \begin{equation*}
           l_i \equiv l_j \to image~p~i < image~p~j
        \end{equation*}
        or
        \begin{equation*}
          (apply~p~l)_i \equiv (apply~p~l)_j \to preimage~p~i < preimage~p~j
        \end{equation*}
      \end{block}
    }
    \only<3-4>{%
      \begin{block}{Inductive Definition: $StablePerm$}
        \setlength{\abovedisplayskip}{\abovedisplayshortskip}
        \setlength{\belowdisplayskip}{\belowdisplayshortskip}
        \begin{equation*}
          \infer[StablePermNil]{StablePerm~[]~[]}{}
        \end{equation*}

        \begin{equation*}
          \infer[StablePermSkip]{StablePerm~(a :: l)~(a :: l')}{%
            StablePerm~l~l'
          }%
        \end{equation*}

        \begin{equation*}
          \infer[StablePermSwap]{%
            \alert<4->{x \not\equiv y \to} StablePerm~(x :: y :: l)~(y :: x :: l)
          }{}
        \end{equation*}

        \begin{equation*}
          \infer[StablePermTrans]{StablePerm~l~l''}{%
            StablePerm~l~l' & StablePerm~l'~l''
          }%
        \end{equation*}
      \end{block}
    }
    \only<5->{
      \begin{block}{Filter Definition: $StableFilter$}
        \setlength{\abovedisplayskip}{\abovedisplayshortskip}
        \setlength{\belowdisplayskip}{\belowdisplayshortskip}
        \begin{equation*}
          \forall x,~filter~(\lambda y. x \equiv y)~l = filter~(\lambda y. x \equiv y)~l'
        \end{equation*}
      \end{block}
    }
  \end{frame}

  \begin{frame}{$StableFilter$ lemmas}
    \setlength{\abovedisplayskip}{\abovedisplayshortskip}
    \setlength{\belowdisplayskip}{\belowdisplayshortskip}
    \begin{block}{Destructuring Lemma}
      \begin{gather*}
        StableFilter~(a :: l)~(b :: l') \to \\
        \exists~l_1~l_2,~l = l_1 \doubleplus b :: l_2 \land (\forall x \in l_1, x \not\equiv b)
      \end{gather*}
    \end{block}
    \begin{block}{Move-to-front Lemma}
      \begin{gather*}
        (\forall x \in l, x \not\equiv b) \to \\
        StableFilter~(l_1 \doubleplus b :: l2)~(b :: l_1 \doubleplus l_2)
      \end{gather*}
    \end{block}
  \end{frame}

  \begin{frame}{Stable Sort Lemmas}
    \setlength{\abovedisplayskip}{\abovedisplayshortskip}
    \setlength{\belowdisplayskip}{\belowdisplayshortskip}
    \begin{block}{Stable Sort Unique}
      \begin{gather*}
        Sorted~l \to Sorted~l' \to StablePerm~l~l' \to l = l'
      \end{gather*}
    \end{block}

    \begin{block}{Leibniz Equality Stable}
      \begin{gather*}
        (\forall~x~y, x \equiv y \to x = y) \to Permutation~l~l' \to StablePerm~l~l'
      \end{gather*}
    \end{block}
  \end{frame}
\end{document}
