\documentclass{beamer}
\usetheme{metropolis}
\usepackage{pgfplots}
\usepackage{ulem}
\usepackage{appendixnumberbeamer}

\usepackage{pgfpages}
\setbeameroption{show notes on second screen=bottom}

\makeatletter
\def\beamer@framenotesbegin{% at beginning of slide
     \usebeamercolor[fg]{normal text}
      \gdef\beamer@noteitems{}%
      \gdef\beamer@notes{}%
}
\makeatother

\title{Towards a Verified Bzip2}
\subtitle{A Functional Specification for Burrows-Wheeler}
\date{\today}
\author{Jake Waksbaum '20\\Advisor: Prof. Andrew Appel}
\institute{Princeton University}

\setmonofont{Iosevka}

\metroset{block=fill}

\pgfplotsset{compat=1.16}
\pgfplotsset{
  barplot/.style={
    %ybar stacked,
    %symbolic x coords={bzip2, 7zip, winRAR, unzip},
    % xtick=data,
    cycle list name=mbarplot cycle,
    axis y line*=none,
    legend style={
      draw=none,
      fill=none,
      cells={anchor=west},
      anchor={north west},
      % at={(0.02,0.98)},
    },
    xmajorgrids=false,
    ymajorgrids=true,
    x tick label style={
      font=\footnotesize
    },
    y tick label style={
      font=\footnotesize
    },
    legend style={
      font=\footnotesize
    },
    major grid style={
      dotted,
    },
    axis x line*=bottom,
  }
}

\begin{document}
  \maketitle

  \subsection{Burrows-Wheeler}

  \begin{frame}{The Burrows-Wheeler Transform}
    \textit{A Block-sorting Lossless Data Compression Algorithm} (1994)
    \begin{itemize}
    \item<+-> A transformation of type \texttt{[a] -> ([a], Nat)}
    \item<+-> Makes it easier to compress
    \item<+-> \textbf{Reversible}
    \end{itemize}
  \end{frame}

  \subsubsection{The Transform}
  \begin{frame}{The Transform}
    Input text: \texttt{abracadabra!}
    \begin{columns}[c] % the "c" option specifies center vertical alignment
      \column{.5\textwidth}
      \begin{enumerate}
      \item<2-> Form rotation matrix
      \item<3-> Sort rotation matrix
      \item<4-> Take the last column of the sorted rotation matrix\ldots
        \only<5->{ and the index where the original string ends up}
      \end{enumerate}

      \column{.5\textwidth}
      \only<2>{%
        \begin{tabular}{rc}
          0  & abracadabra! \\
          1  & bracadabra!a \\
          2  & racadabra!ab \\
          3  & acadabra!abr \\
          4  & cadabra!abra \\
          5  & adabra!abrac \\
          6  & dabra!abraca \\
          7  & abra!abracad \\
          8  & bra!abracada \\
          9  & ra!abracadab \\
          10 & a!abracadabr \\
          11 & !abracadabra
        \end{tabular}
      }%
      \only<3->{%
        \begin{tabular}{rc}
          0  & !abracadabr\alert<4->{a} \\
          1  & a!abracadab\alert<4->{r} \\
          2  & abra!abraca\alert<4->{d} \\
          \textbf<5->{3} & \textbf<5->{abracadabra\alert<4->{!}} \\
          4  & acadabra!ab\alert<4->{r} \\
          5  & adabra!abra\alert<4->{c} \\
          6  & bra!abracad\alert<4->{a} \\
          7  & bracadabra!\alert<4->{a} \\
          8  & cadabra!abr\alert<4->{a} \\
          9  & dabra!abrac\alert<4->{a} \\
          10 & ra!abracada\alert<4->{b} \\
          11 & racadabra!a\alert<4->{b}
        \end{tabular}
      }%
    \end{columns}
  \end{frame}

  \section{Implementation}
  % Describe the steps you have completed and/or the subproblems you
  % have solved to make progress on your project. For each of them,
  % you may want to describe: What was the main issue? What options
  % were available? What solution did you create/choose and why? How
  % did you perform the implementation? What tools did you use? How
  % well does your solution work? What is remaining to be done? If
  % your implementation is not complete, outline any logistical or
  % technical problems you anticipate and explain any contingency plan
  % you have for avoiding or coping with them.
  \begin{frame}[fragile]{Forwards Transform}
    \begin{semiverbatim}
\uncover<+->{lrot, rrot : [a] -> [a]}
\uncover<+->{lrot xs = tl xs ++ [hd xs]}
\uncover<+->{rrot xs = last xs :: init xs}

\uncover<+->{rots : [a] -> [[a]]}
\uncover<+->{rots xs = iter lrot (length xs) xs}

\uncover<+->{bwl : [a] -> [a]}
\uncover<+->{bwl = map last ∘ sort ∘ rots}

\uncover<+->{bwn : [a] -> Nat}
\uncover<+->{bwn xs = findIndex xs $ sort (rots xs)}
    \end{semiverbatim}
  \end{frame}

  \begin{frame}[fragile]{Properties of the Forwards Transform}
    \begin{block}{\texttt{bwl} preserves length}
      \texttt{length ∘ bwl = length}
    \end{block}

    \begin{block}{\texttt{bwn} correct}
      \texttt{sort (rots xs) !! (bwn xs) = xs}
    \end{block}
  \end{frame}

  \begin{frame}[fragile]{Inverse Transform}
    \begin{semiverbatim}
\uncover<1->{unbwt : Nat -> [a] -> [a]}
\uncover<3->{unbwt t xs = \alert{recreate} xs !! t}
    \end{semiverbatim}

    \uncover<2->{%
      \begin{block}{\texttt{unbwt} correct}
        \texttt{unbwt (bwn xs) (bwl xs) = xs}
      \end{block}
    }%
  \end{frame}

  \begin{frame}[fragile]{\texttt{recreate}, recursively}
    \begin{block}{\texttt{recreate} correct}
      \texttt{%
        recreate \only<2->{j }∘ bwl
        = recreate \only<2->{j }∘ map last ∘ sort ∘ rots\\
        = \only<2->{cols j ∘ }sort ∘ rots
      }%
    \end{block}
  \end{frame}

  \begin{frame}[fragile]{\texttt{recreate}, Base case}
    \uncover<+->{\texttt{%
      \alert<+>{cols 0} ∘ sort ∘ rots \\
    }}
    \uncover<.->{\texttt{%
      = map (const []) ∘ sort ∘ rots \\
    }}
    \uncover<+->{\texttt{%
      = \textbf<+->{map (const [])} ∘ bwl \\
    }}
    \uncover<.->{\texttt{%
      = recreate 0 ∘ bwl
    }}
  \end{frame}

  \begin{frame}[fragile]{\texttt{recreate}, Inductive case}
    \uncover<+->{\texttt{%
        cols (j+1) ∘ \alert<+>{sort ∘ rots} \\
    }}
    \uncover<.->{\texttt{%
        = \alert<+>{cols (j+1) ∘ hdsort} ∘ map rrot ∘ sort ∘ rots \\
    }}
    \uncover<.->{\texttt{%
        = hdsort ∘ \alert<+>{cols (j+1) ∘ map rrot} ∘ sort ∘ rots \\
    }}
    \uncover<.->{\texttt{%
        = hdsort ∘ ppCol ∘ fork (\alert<+>{map last}, \alert<+>{cols j})
        \alert<.(-1),.>{∘ sort ∘ rots} \\
    }}
    \uncover<.(-1)->{\texttt{%
        = hdsort ∘ ppCol ∘ fork (bwl, \uncover<.->{recreate j ∘ bwl}) \\
    }}
    \uncover<+->{\texttt{%
        = \textbf<+->{hdsort ∘ ppCol ∘ fork (id, recreate j)} ∘ bwl \\
    }}
    \uncover<.->{\texttt{%
        = recreate (j+1) ∘ bwl
    }}
  \end{frame}

  \section{Results}
  % Every project should have some means of measuring success. Explain
  % the methods you will use to evaluate how well your implementation
  % achieves the goal articulated at the beginning of the talk.
  % Perhaps you will need to describe your test data sets, your
  % measurement techniques, your evaluation metrics, etc. If possible,
  % provide some quantitative comparisons of your results to
  % alternative methods (e.g., the previous state-of-the-art, random
  % results, etc.). Part of (not the entire presentation) may involve
  % giving a demo of your results.
  \begin{frame}{Results}
    QED! \uncover<2->{(Almost)}
    \begin{itemize}
    \item<2-> Some lemmas unproven
    \item<3-> Stability is hard to reason about
    \item<4-> Empty lists are annoying
    \end{itemize}
  \end{frame}

  \section{Future Work}
  \begin{frame}{Future Work}
    \begin{itemize}
      \item<+-> Finish proving lemmas
      \item<+-> Prove more optimal implementations
      \item<+-> Add tools in Coq for working with chains of composed functions
      \item<+-> Explore Birds' relational specifications for program calculation
    \end{itemize}
  \end{frame}

  \appendix

  \section{Motivation \& Goal}
  % Give a high-level introduction to your topic area and state
  % specifically what problem you will be addressing. You should clearly
  % state the goal of your project (i.e., “The goal of my project is
  % …”). Explain why that goal is important and/or interesting. What
  % people and/or applications would benefit? Giving some examples of
  % the problem being addressed may help make it easier to understand.
  \begin{frame}{Compression Software}
    \note[item]{Motivation comes from security and compression software.}
    \note[item]{Compression software reduces the amount of space it takes to
      store information, like a ZIP file.}
    \begin{itemize}
    \item<+-> Reduces the amount of disk space used by files
    \item<+-> Often used to transfer files over the internet
    \item<+-> \alert{Source of security vulnerabilities}
    \end{itemize}
  \end{frame}

  \begin{frame}{Effect of Vulnerabilities in Compression Software}
    \begin{figure}
      \begin{tikzpicture}
        \begin{axis}[
            ybar stacked,
            barplot,
            symbolic x coords={bzip2, 7zip, winRAR, unzip},
            xtick=data,
            legend style={at={(0.02,0.98)}},
          ]
          \addplot+[ybar] plot coordinates {
            (bzip2,  2)
            (7zip,   5)
            (winRAR, 7)
            (unzip,  7)
          };
          \addplot+[ybar] plot coordinates {
            (bzip2,  3)
            (7zip,   1)
            (winRAR, 1)
            (unzip,  6)
          };
          \addplot+[ybar] plot coordinates {
            (bzip2,  4)
            (7zip,   3)
            (winRAR, 5)
            (unzip,  5)
          };
          \legend{\strut Code Exec, \strut DoS, \strut Other}
        \end{axis}
      \end{tikzpicture}
    \end{figure}
  \end{frame}

  \begin{frame}{Cause of Vulnerabilities in Compression Software}
    \begin{figure}
      \begin{tikzpicture}
        \begin{axis}[
            ybar stacked,
            barplot,
            symbolic x coords={bzip2, 7zip, winRAR, unzip},
            xtick=data,
            legend style={at={(0.02,0.98)}},
          ]
          \addplot+[ybar] plot coordinates {
            (bzip2,  2 )
            (7zip,   5 )
            (winRAR, 7 )
            (unzip,  11)
          };
          \addplot+[ybar] plot coordinates {
            (bzip2,  2)
            (7zip,   0)
            (winRAR, 0)
            (unzip,  1)
          };
          \addplot+[ybar] plot coordinates {
            (bzip2,  5)
            (7zip,   4)
            (winRAR, 6)
            (unzip,  6)
          };
          \legend{\strut Mem. Management, \strut Race Cond., \strut Other}
        \end{axis}
      \end{tikzpicture}
    \end{figure}
  \end{frame}

  \begin{frame}{Solution}
    \alt<1>{{\Large\bf Be Smarter}}{\xout{{\Large\bf Be Smarter}}}
    \only<2->{{\Large\bf Formal Methods}}
    \begin{itemize}
    \item<2-> Use math (checked by computers) to verify correctnesss
    \end{itemize}
  \end{frame}

  \begin{frame}{Goal}
    The \textbf{goals} of my work are:
    \begin{itemize}
    \item<1-> to lay the groundwork for a verified implementation of
      bzip2
      \begin{itemize}
      \item<2-> a \textit{functional specification} for its main
        algorithm, the Burrows-Wheeler transform
      \end{itemize}
    \item<3-> to find techniques that are useful in general for
      machine-checked proofs in Coq
    \end{itemize}
  \end{frame}

  \section{Problem Background \& Related Work}
  % You probably are not the first person to work towards this goal,
  % nor will you be the last. Give some background of what has been
  % done before. Have academics already written papers on this topic?
  % Are there already products on the market? Why do they not solve
  % the problem? What aspects of the problem are still uninvestigated?
  % It is always the case that someone has done something that can be
  % related to what you are going to do. Find out the most closely
  % related pieces of work and explain the relationship to your
  % proposed project.

  \begin{frame}{Distribution of MTF Positions after BWT}
    \begin{figure}
      \begin{tikzpicture}
        \begin{axis}[
            ybar,
            barplot,
            xlabel={MTF Position},
            xtick=data,
            yticklabel={\pgfmathprintnumber\tick\,\%},
          ]
          \addplot+[ybar] plot coordinates {
            (0, 66.8)
            (1, 9.0)
            (2, 4.0)
            (3, 2.9)
            (4, 2.3)
            (5, 1.8)
            (6, 1.6)
            (7, 1.4)
          };
        \end{axis}
      \end{tikzpicture}
    \end{figure}
  \end{frame}

  \subsubsection{Coq \& VST}
  \begin{frame}{Coq \& VST}
    \begin{description}
      \item[Coq] A theorem proving assistant
      \item[CompCert] An optimizing C compiler formally verified in
        Coq
      \item[VST] Program logic for C
    \end{description}
  \end{frame}

  \subsubsection{Program Calculation}
  \begin{frame}{Program Calculation}
    \begin{itemize}
    \item<+-> A technique for deriving programs from specifications using
      \alert{equational reasoning}
    \item<+-> Programs are \alert{correct by construction}
    \end{itemize}
  \end{frame}

  \begin{frame}[fragile]{Program Calculation -- An Example}
    \begin{semiverbatim}
maximum = hd ∘ sort

\uncover<+->{maximum [a]} \uncover<+->{= hd $ sort [a]}
\uncover<+->{= hd [a]}
\uncover<+->{= a}

\uncover<+->{maximum (a :: x)} \uncover<+->{= hd $ sort (a :: x)}
\uncover<+->{= hd $ if a > hd (sort x)
       then a :: sort x
       else hd (sort x) :: insert a (tail (sort x))}
\uncover<+->{= if a > hd (sort x) then a else hd (sort x)}
\uncover<+->{= if a > maximum x then a else maximum x}
\uncover<+->{= max a (maximum x)}
    \end{semiverbatim}
  \end{frame}

  \section{Approach}
  %  Clearly explain the key idea behind the approach you are taking
  %  and explain why it is a good idea. What is your key insight to
  %  solve the problem? What makes your approach unique in comparison
  %  to previous work? It might be that you are asking a different
  %  question than others have before, or attacking the problem in a
  %  different way, or using different tools, or leveraging different
  %  data sets. In any case, describe the most interesting “key idea”
  %  behind your project and justify why you chose it.
  \begin{frame}{Approach}
    Use \textbf{VST} to produce a verifized implementation of bzip2
    \begin{itemize}
    \item Define a functional specification and prove ``correct''
      \begin{itemize}
      \item \alert{Burrows-Wheeler}
      \item Move-to-Front
      \item Run-length Encoding
      \item Huffman Encoding
      \end{itemize}
    \item Write program in C and prove equivalent to specification
    \end{itemize}

    \visible<2->{Use \textbf{program calculation} to write specifications and
      prove equivalence between implementations}
  \end{frame}

  \subsubsection{Why it works}
  \begin{frame}{Why it works}
    \texttt{transform abracadabra! = (ard!rcaaaabb, 7)}
    \begin{itemize}
      \item<+-> Groups characters based on context
      \item<+-> After MTF, perfect for RLE and Huffman
    \end{itemize}
  \end{frame}


\end{document}
