\documentclass{beamer}
\usetheme{metropolis}
\usepackage{pgfplots}
\usepackage{ulem}

\title{Towards a Verified Bzip2}
\subtitle{A Functional Specification for Burrows-Wheeler}
\date{\today}
\author{Jake Waksbaum '20\\Advisor: Prof. Andrew Appel}
\institute{Princeton University}

\setmonofont{Iosevka}

\begin{document}
  \maketitle

  \section{Motivation \& Goal}
  % Give a high-level introduction to your topic area and state
  % specifically what problem you will be addressing. You should clearly
  % state the goal of your project (i.e., “The goal of my project is
  % …”). Explain why that goal is important and/or interesting. What
  % people and/or applications would benefit? Giving some examples of
  % the problem being addressed may help make it easier to understand.
  \begin{frame}{Compression Software}
    \begin{itemize}
    \item<+-> Reduces the amount of disk space used by files
    \item<+-> Often used to transfer files over the internet
    \item<+-> \alert{Source of security vulnerabilities}
    \end{itemize}
  \end{frame}

  \begin{frame}{Effect of Vulnerabilities in Compression Software}
    \begin{figure}
      \begin{tikzpicture}
        \begin{axis}[
            ybar stacked,
            symbolic x coords={bzip2, 7zip, winRAR, unzip},
            xtick=data,
            cycle list name=mbarplot cycle,
            axis y line*=none,
            legend style={
              draw=none,
              fill=none,
              cells={anchor=west},
              anchor={north west},
              at={(0.02,0.98)},
            },
            xmajorgrids=false,
            ymajorgrids=true,
            x tick label style={
              font=\footnotesize
            },
            y tick label style={
              font=\footnotesize
            },
            legend style={
              font=\footnotesize
            },
            major grid style={
              dotted,
            },
            axis x line*=bottom,
          ]
          \addplot+[ybar] plot coordinates {
            (bzip2,  2)
            (7zip,   5)
            (winRAR, 7)
            (unzip,  7)
          };
          \addplot+[ybar] plot coordinates {
            (bzip2,  3)
            (7zip,   1)
            (winRAR, 1)
            (unzip,  6)
          };
          \addplot+[ybar] plot coordinates {
            (bzip2,  4)
            (7zip,   3)
            (winRAR, 5)
            (unzip,  5)
          };
          \legend{\strut Code Exec, \strut DoS, \strut Other}
        \end{axis}
      \end{tikzpicture}
    \end{figure}
  \end{frame}

  \begin{frame}{Cause of Vulnerabilities in Compression Software}
    \begin{figure}
      \begin{tikzpicture}
        \begin{axis}[
            ybar stacked,
            symbolic x coords={bzip2, 7zip, winRAR, unzip},
            xtick=data,
            cycle list name=mbarplot cycle,
            axis y line*=none,
            legend style={
              draw=none,
              fill=none,
              cells={anchor=west},
              anchor={north west},
              at={(0.02,0.98)},
            },
            xmajorgrids=false,
            ymajorgrids=true,
            x tick label style={
              font=\footnotesize
            },
            y tick label style={
              font=\footnotesize
            },
            legend style={
              font=\footnotesize
            },
            major grid style={
              dotted,
            },
            axis x line*=bottom,
          ]
          \addplot+[ybar] plot coordinates {
            (bzip2,  2 )
            (7zip,   5 )
            (winRAR, 7 )
            (unzip,  11)
          };
          \addplot+[ybar] plot coordinates {
            (bzip2,  2)
            (7zip,   0)
            (winRAR, 0)
            (unzip,  1)
          };
          \addplot+[ybar] plot coordinates {
            (bzip2,  5)
            (7zip,   4)
            (winRAR, 6)
            (unzip,  6)
          };
          \legend{\strut Mem. Management, \strut Race Cond., \strut Other}
        \end{axis}
      \end{tikzpicture}
    \end{figure}
  \end{frame}

  \begin{frame}{Solution}
    \alt<1>{{\Large\bf Be Smarter}}{\xout{{\Large\bf Be Smarter}}}
    \only<2->{{\Large\bf Formal Methods}}
    \begin{itemize}
    \item<2-> Use math (checked by computers) to verify correctnesss
    \end{itemize}
  \end{frame}

  \begin{frame}{Goal}
    The \textbf{goals} of my work are:
    \begin{itemize}
    \item<1-> to lay the groundwork for a verified implementation of
      bzip2
      \begin{itemize}
      \item<2-> a \textit{functional specification} for its main
        algorithm, the Burrows-Wheeler transform
      \end{itemize}
    \item<3-> to find techniques that are useful in general for
      machine-checked proofs in Coq
    \end{itemize}
  \end{frame}

  \section{Problem Background \& Related Work}
  % You probably are not the first person to work towards this goal,
  % nor will you be the last. Give some background of what has been
  % done before. Have academics already written papers on this topic?
  % Are there already products on the market? Why do they not solve
  % the problem? What aspects of the problem are still uninvestigated?
  % It is always the case that someone has done something that can be
  % related to what you are going to do. Find out the most closely
  % related pieces of work and explain the relationship to your
  % proposed project.

  \subsection{Burrows-Wheeler}

  \tableofcontents[
    sectionstyle=show/shaded,
    subsectionstyle=show/shaded,
    subsubsectionstyle=show/show/hide/hide,
  ]

  \begin{frame}{The Burrows-Wheeler Transform}
    \textit{A Block-sorting Lossless Data Compression Algorithm} (1994)
    \begin{itemize}
    \item<1-> A transformation of type \texttt{Ord a => [a] -> ([a], Nat)}
    \item<2-> Makes it easier to compress
    \item<4-> \textbf{Reversible}
    \end{itemize}
  \end{frame}

  \subsubsection{The Transform}
  \begin{frame}{The Transform}
    Input text: \texttt{kaukapakapa}
    \begin{columns}[c] % the "c" option specifies center vertical alignment
      \column{.5\textwidth}
      \begin{enumerate}
      \item<2-> Form suffix matrix
      \item<3-> Sort
      \item<4-> Take the last column of the sorted suffix matrix\ldots
        \only<5->{ and the index where the original string ends up}
      \end{enumerate}

      \column{.5\textwidth}
      \only<2>{%
        \begin{tabular}{rc}
          0  & kaukapakapa \\
          1  & aukapakapak \\
          2  & ukapakapaka \\
          3  & kapakapakau \\
          4  & apakapakauk \\
          5  & pakapakauka \\
          6  & akapakaukap \\
          7  & kapakaukapa \\
          8  & apakaukapak \\
          9  & pakaukapaka \\
          10 & akaukapakap
        \end{tabular}
      }%
      \only<3>{%
        \begin{tabular}{rc}
          0  & akapakaukap \\
          1  & akaukapakap \\
          2  & apakapakauk \\
          3  & apakaukapak \\
          4  & aukapakapak \\
          5  & kapakapakau \\
          6  & kapakaukapa \\
          7  & kaukapakapa \\
          8  & pakapakauka \\
          9  & pakaukapaka \\
          10 & ukapakapaka
        \end{tabular}
      }%
      \only<4>{%
        \begin{tabular}{rc}
          0  & akapakauka\alert{p} \\
          1  & akaukapaka\alert{p} \\
          2  & apakapakau\alert{k} \\
          3  & apakaukapa\alert{k} \\
          4  & aukapakapa\alert{k} \\
          5  & kapakapaka\alert{u} \\
          6  & kapakaukap\alert{a} \\
          7  & kaukapakap\alert{a} \\
          8  & pakapakauk\alert{a} \\
          9  & pakaukapak\alert{a} \\
          10 & ukapakapak\alert{a}
        \end{tabular}
      }%
      \only<5>{%
        \begin{tabular}{rc}
          0         & akapakauka\alert{p}          \\
          1         & akaukapaka\alert{p}          \\
          2         & apakapakau\alert{k}          \\
          3         & apakaukapa\alert{k}          \\
          4         & aukapakapa\alert{k}          \\
          5         & kapakapaka\alert{u}          \\
          6         & kapakaukap\alert{a}          \\
          \alert{7} & \textbf{kaukapakap\alert{a}} \\
          8         & pakapakauk\alert{a}          \\
          9         & pakaukapak\alert{a}          \\
          10        & ukapakapak\alert{a}
        \end{tabular}
      }%
    \end{columns}
  \end{frame}

  \subsubsection{Why it works}
  \begin{frame}[fragile]{Why it works}
    \texttt{transform kaukapakapa = (ppkkkuaaaaa, 7)}

    \only<1-2>{%
    \begin{itemize}
      \item<1-2> Groups characters based on context
      \item<2-2> After MTF, perfect for RLE and Huffman
    \end{itemize}
    }%

    \only<3->,
            ]
            \addplot plot coordinates {
              (0, 66.8)
              (1, 9.0)
              (2, 4.0)
              (3, 2.9)
              (4, 2.3)
              (5, 1.8)
              (6, 1.6)
              (7, 1.4)
            };
          \end{axis}
        \end{tikzpicture}
      \end{figure}
    }%
  \end{frame}

  \subsection{Verifying Programs}
  \tableofcontents[
    sectionstyle=show/shaded,
    subsectionstyle=show/shaded,
    subsubsectionstyle=show/show/hide/hide,
  ]

  \subsubsection{Coq \& VST}
  \begin{frame}{Coq \& VST}
    \begin{description}
      \item[Coq] A theorem proving assistant
      \item[CompCert] An optimizing C formally verified in Coq
      \item[VST] Program logic for C
    \end{description}
  \end{frame}

  \subsubsection{Program Calculation}
  \begin{frame}{Program Calculation}
    \begin{itemize}
    \item<+-> A technique for deriving programs from specifications using
      \alert{equational reasoning}
    \item<+-> Programs are \alert{correct by construction}
    \end{itemize}
  \end{frame}

  \section{Approach}
  %  Clearly explain the key idea behind the approach you are taking
  %  and explain why it is a good idea. What is your key insight to
  %  solve the problem? What makes your approach unique in comparison
  %  to previous work? It might be that you are asking a different
  %  question than others have before, or attacking the problem in a
  %  different way, or using different tools, or leveraging different
  %  data sets. In any case, describe the most interesting “key idea”
  %  behind your project and justify why you chose it.
  \begin{frame}{Approach}
    Use \textbf{VST} to produce a verifized implementation of bzip2
    \begin{itemize}
    \item Define a functional specification and prove ``correct''
      \begin{itemize}
      \item \alert{Burrows-Wheeler}
      \item Move-to-Front
      \item Run-length Encoding
      \item Huffman Encoding
      \end{itemize}
    \item Write program in C and prove equivalent to specification
    \end{itemize}

    \visible<2->{Use \textbf{program calculation} to write specifications and
      prove equivalence between implementations}
  \end{frame}

  \section{Implementation}
  % Describe the steps you have completed and/or the subproblems you
  % have solved to make progress on your project. For each of them,
  % you may want to describe: What was the main issue? What options
  % were available? What solution did you create/choose and why? How
  % did you perform the implementation? What tools did you use? How
  % well does your solution work? What is remaining to be done? If
  % your implementation is not complete, outline any logistical or
  % technical problems you anticipate and explain any contingency plan
  % you have for avoiding or coping with them.
  \begin{frame}[fragile]{Forwards Transform}
    \begin{semiverbatim}
transform    :: Ord a => [a] -> ([a], Nat)
transform xs = (map last mat, findIndex xs mat)
               where mat = sort (rots xs)

rots    :: [a] -> [[a]]
rots xs = iter lrot (length xs) xs

lrot, rrot :: [a] -> [a]
lrot xs = tl xs ++ hd xs
rrot xs = last xs : init xs
    \end{semiverbatim}
  \end{frame}

  \begin{frame}[fragile]{Inverse Transform}
    \begin{semiverbatim}
untransform         :: Ord a => ([a], Nat) -> [a]
\alt<1>{untransform ∘ transform = id}{untransform (lc, i) = (recreate lc) !! i}

\visible<2->{recreate :: Ord a => [a] -> [[a]]}
\only<2>{recreate ∘ fst ∘ transform = sort ∘ rots}\only<3->{recreate ∘ map last ∘ sort ∘ rots = sort ∘ rots}
    \end{semiverbatim}
  \end{frame}

  \begin{frame}[fragile]{\texttt{recreate}, recursively}
    \begin{semiverbatim}
\alt<2->{recreate ∘ map last ∘ sort ∘ rots = sort ∘ rots}{recreate j ∘ map last ∘ sort ∘ rots = cols j ∘ sort ∘ rots}
    \end{semiverbatim}
  \end{frame}

  \begin{frame}[fragile]{\texttt{recreate}, Base case}
    \begin{semiverbatim}
      recreate 0 = map (const [])
    \end{semiverbatim}
  \end{frame}

  \begin{frame}[fragile]{\texttt{recreate}, Inductive case}
    \begin{semiverbatim}
\alert<8->{recreate (j+1)} ∘ map last ∘ sort ∘ rots
\uncover<2->{= cols (j+1) ∘ sort ∘ rots}
\uncover<3->{= hdsort ∘ map rrot ∘ cols (j+1) ∘ sort ∘ rots}
\uncover<4->{= cols (j+1) ∘ hdsort ∘ map rrot ∘ sort ∘ rots}
\uncover<5->{= hdsort ∘ cols (j+1) ∘ map rrot ∘ sort ∘ rots}
\uncover<6->{= hdsort ∘ consCol ∘ fork (map last, recreate j ∘ map last)
  ∘ sort ∘ rots}
\uncover<7->{= \alert<8->{hdsort ∘ consCol ∘ fork (id, recreate j)}
  ∘ map last ∘ sort ∘ rots}
    \end{semiverbatim}
  \end{frame}

  \section{Results}
  % Every project should have some means of measuring success. Explain
  % the methods you will use to evaluate how well your implementation
  % achieves the goal articulated at the beginning of the talk.
  % Perhaps you will need to describe your test data sets, your
  % measurement techniques, your evaluation metrics, etc. If possible,
  % provide some quantitative comparisons of your results to
  % alternative methods (e.g., the previous state-of-the-art, random
  % results, etc.). Part of (not the entire presentation) may involve
  % giving a demo of your results.
  \begin{frame}{Results}
    \begin{itemize}
    \item QED! (Almost)
    \item Some lemmas are still unproven
      \begin{itemize}
      \item Stability is hard to reason about
      \item Empty lists are annoying
      \end{itemize}
    \end{itemize}
  \end{frame}

  \section{Future Work}
  \begin{frame}{Future Work}
    \begin{itemize}
      \item Finish proving lemmas
      \item Add tools in Coq for working with chains of composed functions
      \item Explore Birds' relational specifications for program calculation
    \end{itemize}
  \end{frame}
\end{document}
